
\documentclass[twocolumn]{article}
%% \usepackage{a4}
\usepackage{palatino}
% \usepackage{euler}
% \usepackage{fancyhdr}

\newcommand {\atilde} {$_{\char '176}$} % tilde(~) character
\setlength{\columnsep}{10mm}
%\textwidth 6.0in % doesn't change left margin
\oddsidemargin 0.1in 

\title{Coot Crib Sheet}
\begin{document}
\maketitle

\section{Keyboard}

\subsection{Dialog Shortcuts}
\begin{tabular}{ll}
  F6  & Post Go To Atom window \\
  F7  & Post Display Control Window\\
\end{tabular}

\subsection{Previous/Next Residue}

\begin{tabular}{ll}
  ``Space'' & Next Residue \\
  ``Shift'' ``Space'' & Previous Residue
\end{tabular}

\subsection{Closest Residue}
``p'' go to an atom of the closest residue (the ``CA'' atom if the
residue has one)

\subsection{Go To Residue}
Ctrl-g $<$\emph{Residue-number}$>$$<$\emph{Enter}$>$

Jump to the give residue (you can provide a chain-id too\footnote
{The chain-id goes directly before the residue number, i.e.
 
Ctrl-g $<$\emph{Chain-id}$>$$<$\emph{Residue-number}$>$$<$\emph{Enter}$>$})

\subsection{Next NCS Chain}
``o'' - other NCS chain.

\subsection{``Undo'' Move}
``u'' to undo the move recent screen recentering (\emph{e.g.} move
back after recentering after reading a new PDB file)

\subsection{Previous/Next Rotamer}

When in ``Rotamer'' mode, these keyboard short-cuts are
available\footnote{note: focus must be in the graphics window, not
  the Rotamer dialog}:

\begin{tabular}{ll}
  ``.'' & Next Rotamer \\
  ``,'' & Previous Rotamer
\end{tabular}

\subsection{Keyboard Chi Angles}
Instead of pressing the buttons in the Chi Angles button box, you can
use keyboard ``1'' for Chi1, ``2'' for Chi2 \emph{etc.}

\subsection{Keyboard Contouring}

Use ``\texttt{+}'' or ``\texttt{-}'' to change the contour level

\subsection{Keyboard Labelling}
``l'' to label closest atom

\subsection{Quick Save As} 
   Ctrl-s to save the state and any unsaved molecules (to default file names).

\subsection{Keyboard Residue Info} 
   Ctrl-i then click on residue to open Residue Info dialog 

\subsection{Keyboard Navigation}
\begin{tabular}{ll}
  Ctrl-G & Type the residue number and chain-id \\
\end{tabular}

\subsection{Keyboard Translation}
\begin{tabular}{ll}
  Keypad 3 & Push View (+Z translation)\\
  Keypad . & Pull View (-Z translation)
\end{tabular}

\subsection{Keyboard Undo/Redo}
\begin{tabular}{ll}

  Ctrl-z & Undo last modification   \\
  Ctrl-y & Redo last modification   \\
  u & Undo last move/navigation     \\
\end{tabular}

\subsection{Editing}
\begin{tabular}{ll}

  Ctrl-c & Copy active molecule   \\
  Ctrl-x & Delete active residue  \\
\end{tabular}
	

\subsection{Keyboard Zoom and Clip}

\begin{tabular}{ll}

  n & Zoom out   \\
  m & Zoom in    \\
  d & Slim clip  \\
  f & Fatten clip\\
\end{tabular}

\subsection{Crosshairs}
c: cross-hairs

\subsection{Skeleton}
s: Generate skeleton around current point\footnote{if a skeleton is being
displayed}

\subsection{Continuous Rotate}
i: Toggle continuous spin

\subsection{Baton Mode}
b: toggle into baton rotate mode\footnote{rather than view rotate
  mode}

\newpage
\section{Mouse}
Mouse actions are occassionally augmented with keyboard modifiers:
  \vspace{5mm}

  \begin{tabular}{ll}
    Left-mouse Drag & Rotate view \\
    Ctrl Left-Mouse Drag &  Translates view \\
    Shift Left-Mouse Click &  Label Atom\\
    Right-Mouse Drag &  Zoom in and out\index{zoom}\\
    Shift Right-Mouse Drag & Change clipping and Translate in \\
                           & Screen Z \\
                           & The movement is along orthogonal \\
                           & axes: \\
                           & up+right/down+left shifts in z, \\ 
                           &  up+left/down+right changes the \\
                           & slab \\
    Ctrl Shift Right-Mouse Drag &  Rotate View about Screen Z\\
    Middle-mouse Click & Centre on atom\\
    Scroll-wheel Forward &  Increase map contour level\\
    Scroll-wheel Backward &  Decrease map contour level
  \end{tabular}

  \vspace{5mm}
  Intermediate (white) atoms can be dragged around by clicking on
  them:

  \vspace{5mm}
\begin{tabular}{ll}
 Left-mouse Drag:     & Move all intermediate \\
                      & atoms by linear shear \\
 Left-mouse Drag  & as above with\\
  with ``A'' key:
                                 &  non-linear shear\\
 Left-mouse Drag & Move a single atom\\
 with ``Ctrl'': 
\end{tabular}

\section{Refinement Extras}
Use ''A'' to define a residue range\footnote{+/- \emph{n} residues
  from the current residue} with a single-click. Useful in Refinement
and Regularization.

\begin{itemize}
\item Click ``Real Space Refine Zone''
\item Click on an atom
\item Press the ``A'' key
\end{itemize}

\section{Template Key-bindings}

\begin{tabular}{ll}

  E & Flip Ligand \\
  G & Go To Blob (under cursor) \\
  H & Neighbour refine \\
  J & Jiggle Fit This Residue \\
  K & Fill Partial Side-chain \\
  R & Refine Active Residue \\
  T & Triple Residue Refine \\
  X & Refine Active Residue and Auto-accept \\
  W & Add Water \\
  Y & Add Terminal Residue \\
  Shift-Q & Rotamer Dialog for Residue \\
  Shift-R & Sphere Refine \\
  Shift A & Accept Baton Position \\
  Shift-B & Sphere Regularize \\
  Shift P & Delete Residue Hydrogens \\
  Shift V & Undo Symmetry View \\
  Shift-X & Edit Chi Angles \\
  Shift-W & Add Water to Blob \\
  Shift 4 & Ball and Stick for Ligand
\end{tabular}


\end{document}

% \begin{trivlist}
% \item Left-mouse Drag: Move all intermediate atoms by linear shear
% \item Left-mouse Drag with ``A'' key: as above with non-linear shear
% \item Left-mouse Drag with ``Ctrl'': Move a single atom
% \end{trivlist}

%   \vspace{5mm}

% \begin{tabular}{ll}
%  Left-mouse Drag: & Move all intermediate atoms by linear shear \\
%  Left-mouse Drag with ``A'' key: & as above with non-linear shear \\
%  Left-mouse Drawg with ``Ctrl'': & Move a single atom
% \end{tabular}
